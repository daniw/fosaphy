% coding:utf-8

%----------------------------------------
%FOSAPHY, a LaTeX-Code for a summary of basic physics
%Copyright (C) 2013, Daniel Winz, Ervin Mazlagic

%This program is free software; you can redistribute it and/or
%modify it under the terms of the GNU General Public License
%as published by the Free Software Foundation; either version 2
%of the License, or (at your option) any later version.

%This program is distributed in the hope that it will be useful,
%but WITHOUT ANY WARRANTY; without even the implied warranty of
%MERCHANTABILITY or FITNESS FOR A PARTICULAR PURPOSE.  See the
%GNU General Public License for more details.
%----------------------------------------

\chapter*{Über diese Arbeit}
Dies ist das Ergebnis einer Zusammenarbeit auf Basis freier Texte erstellt von Studierenden der Fachhochschule Luzern und ist unter der GPLv2 lizenziert. Der \TeX - bzw. \LaTeX -Code ist auf \url{github.com/daniw/fosaet} hinterlegt.Eine aktuelle PDF-Ausgabe steht auf \url{fosa.adinox.ch} zum Download bereit.

In dieser Formelsammlung sind die Inhalte des Physikteil des Moduls Ma+PHY1T der HSLU-T\&A zusammengefasst. 

Allfällige Fehler können per E-Mail an die Autoren 
(\href{mailto:nino.ninux@gmail.com}{\nolinkurl{nino.ninux@gmail.com}} oder 
\href{mailto:daniel.winz@stud.hslu.ch}{\nolinkurl{daniel.winz@stud.hslu.ch}}) 
gemeldet werden. 
