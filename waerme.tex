% coding:utf-8

%----------------------------------------
%FOSAPHY, a LaTeX-Code for a summary of basic physics
%Copyright (C) 2013, Daniel Winz, Ervin Mazlagic

%This program is free software; you can redistribute it and/or
%modify it under the terms of the GNU General Public License
%as published by the Free Software Foundation; either version 2
%of the License, or (at your option) any later version.

%This program is distributed in the hope that it will be useful,
%but WITHOUT ANY WARRANTY; without even the implied warranty of
%MERCHANTABILITY or FITNESS FOR A PARTICULAR PURPOSE.  See the
%GNU General Public License for more details.
%----------------------------------------

\chapter{Wärme}

\section{Gasgleichung}
\[ \boxed{\frac{p \cdot V}{\vartheta \cdot n} = const} \]
\[ \boxed{p \cdot V = N \cdot k \cdot \vartheta} \]
\[ \boxed{p \cdot V = n \cdot R \cdot \vartheta} \]
$k = 1.38 \cdot 10^{-23} \frac{K}{J}$ \\
$R = 8.31 \frac{J}{mol \cdot K}$ \\
$N$: Anzahl Teilchen \\
$n$: Stoffmenge in Mol \\
$N_A = 6.022 \cdot 10^{23} \frac{1}{mol}$ (Avogadrozahl) \\
\[ n = \frac{N}{N_A} = \frac{m}{M_{mol}} \]
\subsection{Molare Masse}
\[ M_{Luft} = 28.8 \frac{g}{mol} \]

\section{Barometrische Höhenformel}
\[ \boxed{p 
= p_0 \cdot e^{- \frac{M_{mol} \cdot g}{R \cdot \vartheta} \cdot y} 
= p_0 \cdot e^{- \frac{y}{H}}} \qquad \text{$\vartheta$ konstant}\]
\[ \boxed{p = p_0 \cdot \left(1 - \frac{b}{a} \cdot y\right)^
{\frac{M_{mol} \cdot g}{R \cdot b}}} \]
$a = 288.15 K$ \\
$b = 6.5 \frac{K}{km}$ \\

\section{Druck}
\[ \boxed{p = \frac{F}{A} = m_{mol} \cdot {v_x}^2 \cdot \frac{N}{V}} \]

\section{Kinetische Gastheorie}
\[ P = \frac{N}{V} \frac{1}{3} m_{kül}\langle v^2\rangle  = \frac{N}{V} \frac{2}{3} \langle E_{kin, kül}\rangle  = \frac{N}{V} k \vartheta  \]
mikroskopisch: 
\[ \langle E_{kin, kül}\rangle  = \frac{1}{2} m_{kül} \langle v^2\rangle  = 3 \cdot \frac{1}{2} k \vartheta \]
makroskopisch: 
\[ E_{Gas} = 3 \cdot n \frac{1}{2} R \vartheta\]
\[ \langle E_{kin, kül}\rangle  \sim \vartheta \]
\[ \langle E_{kin, kül}\rangle  = 3 \frac{1}{2} k \vartheta \]

\section{Spezifische Wärmekapazität}
\[ Q = m \cdot c \cdot \Delta \vartheta \]
\[ dQ = m \cdot c \cdot d \vartheta \]
\[ c_{(m)} = \frac{1}{m} \frac{dQ}{d \vartheta} \qquad \text{c pro Masse}\]
\[ c_{(n)} = \frac{1}{n} \frac{dQ}{d \vartheta} \qquad \text{c pro Mol}\]

\subsection{Spezifische Wärmekapazität von Gasen}
konstantes Gasvolumen
\[ c_{(n)V} = C_{V} = \frac{\#FG}{2}R \]
\begin{tabular}{lll}
                                      & \#FG     & $C_V$ \\
Ideale Gase: ($Ar$, $Ne$, $He$, ...)  & $3$      & $\frac{3}{2} R$ \\
zwei-atomige: ($O_2$, $N_2$, CO, ...) & $\sim 5$ & $\frac{5}{2} R$ 
\end{tabular}

\section{Mittlere freie Weglänge}
\[ \boxed{t_{mean} = \frac{d t}{d N_{Stoss}} 
= \frac{V}{4 \cdot \pi \cdot \sqrt{2} \cdot r^2 \cdot v \cdot N}} \]
\[ \boxed{\Lambda = v \cdot t_{mean} 
= \frac{V}{4 \cdot \pi \cdot \sqrt{2} \cdot r^2 \cdot N} 
= \frac{k_b \cdot T}{4 \cdot \pi \cdot \sqrt{2} \cdot r^2 \cdot p}} \]
\[ \boxed{v_{rms} = \sqrt{\frac{3 \cdot R \cdot T}{M_{mol}}}} \]

\section{Maxwell-Boltzmann Verteilung}
\[ \boxed{f(v) = 4 \pi \left(\frac{m_{mol}}{2 \pi \cdot R \cdot T}\right)^{\frac{3}{2}}
\cdot v^2 \cdot e^{-\left(\frac{m_{mol} \cdot v^2}{2 \cdot R \cdot T}\right)}} \]
\[ \boxed{f(v) = 4 \pi \left(\frac{m_{kül}}{2 \pi \cdot k_B \cdot T}\right)^{\frac{3}{2}}
\cdot v^2 \cdot e^{-\left(\frac{m_{kül} \cdot v^2}{2 \cdot k_B \cdot T}\right)}} \]

\section{Charakteristische Geschwindigkeiten}
Wahrscheinlichste Geschwindigkeit
\[ \boxed{v_w = \sqrt{\frac{2 \cdot k_B \cdot T}{m_{kül}}} 
= \sqrt{\frac{2 \cdot R \cdot T}{m_{mol}}}} \]
Durchschnittsgeschwindigkeit
\[ \boxed{v_{av} = \sqrt{\frac{8 \cdot k_b \cdot T}{\pi \cdot m_{kül}}} 
= \sqrt{\frac{8 \cdot R \cdot T}{\pi \cdot m_{mol}}}} \]
Root mean Square Geschwindigkeit
\[ \boxed{v_w = \sqrt{\frac{3 \cdot k_B \cdot T}{m_{kül}}} 
= \sqrt{\frac{3 \cdot R \cdot T}{m_{mol}}}} \]

\section{Volumenarbeit eines Gases}
\[ \boxed{W = -\int_{V_1}^{V_2} p \cdot dV = - p \cdot (V_2 - V_1)} \]

\section{Zustandsänderungen des idealen Gas}

\subsubsection{adiabatischer Prozess}
kein Wärmeaustausch
\[ d Q = 0 \]
\[ Q_{a1} = 0 \]
\[ \Delta U_{a1} = n \cdot C_v \cdot (T_1 - T_a) = W_{a1} 
= \frac{p_2 \cdot V_2 - p_1 \cdot V_1}{\gamma - 1} \]
\[ T_1 \cdot {V_1}^{(\gamma - 1)} = T_2 \cdot {V_2}^{(\gamma - 1)} \]
\[ p_1 \cdot {V_1}^{\gamma} = p_2 \cdot {V_2}^{\gamma} \]
\[ T_1 \cdot {p_1}^{\left(\frac{1 - \gamma}{\gamma}\right)} 
= T_2 \cdot {p_2}^{\left(\frac{1 - \gamma}{\gamma}\right)} \]

\subsubsection{isochorer Prozess}
Keine Volumenänderung
\[ \Delta V = 0 \]
\[ \Delta W_{a2} = 0 \]
\[ \Delta Q_{a2} = n \cdot C_v \cdot (T_2 - T_a) = \Delta U_{a2} \]

\subsubsection{isobarer Prozess}
Keine Druckänderung
\[ \Delta p = 0 \]
\[ \Delta U_{a3} = n \cdot C_v \cdot (T_3 - T_a) \]
\[ \Delta W_{a3} = - p \cdot (V_3 - V_a) = - n \cdot R \cdot (T_3 - T_a) \]
\[ \Delta Q_{a3} = n \cdot C_p \cdot (T_3 - T_a)\]

\subsubsection{isothermer Prozess}
Keine Temperaturänderung
\[ \Delta T = 0 \]
\[ \Delta U_{a4} = 0 \]
\[ W_{a4} = \int_{v_a}^{v_4} - p dV = - n \cdot R \cdot T \cdot 
\ln\left(\frac{V_4}{V_a}\right) = - Q_{a4} \]

\section{Spezifische Wärme $C_p$ des idealen Gas}
\[ C_p = C_v + R \]
\[ \gamma \equiv \frac{C_p}{C_v} = \frac{C_v + R}{C_v} \]
\[ \gamma = 1.67 \qquad \text 1 atomiges Gas \]
\[ \gamma = 1.40 \qquad \text 2 atomiges Gas \]