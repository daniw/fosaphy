\section{Gedämpfte Schwingung}
Nimmt die Amplitude einer Schwingung im Verlaufe der Zeit ab, so nennt 
man diese gedämpft. Um ein einfach harmonisches Schwingungssystem mit 
einer Dämpfung zu erweitern, bedarf es lediglich der Erweiterung der
ursprünglichen Differentialgleichung.
\[ \boxed{\vec{F} 
	= m \cdot \ddot{x} + \underbrace{
		b \cdot \dot{x}}_{\text{Dämpfung}} 
	+ k \cdot x = 0
} \]



\[ \boxed{F_{\text{Dämpfung}} = F_{\text{Stokes}} 
= 6 \pi \cdot \eta \cdot R \cdot v} \]
\[ \boxed{\rightarrow b = 6 \pi \cdot \eta \cdot R} \]
\[ \boxed{\omega = \sqrt{\frac{k}{m}}} \]
\[ \boxed{\beta = \frac{b}{2 m}} \]

Die Abklingkonstante $\beta$ kann je nach Wert zu drei Fällen der
Schwingung führen.
\begin{itemize}
	\item $\beta > \omega$ \hfill{} Kriechfall
	\item $\beta = \omega$ \hfill{} kritische Dämpfung
	\item $\beta < \omega$ \hfill{} gedämpfte Schwingung
\end{itemize}

\subsection{Kriechfall}
Im Kriechfall ist die Abklingkonstante $\beta$ grösser als die 
Kreisfrequenz $\omega$. Dies bedeutet, dass gar keine Schwingung
mehr zu Stande kommt und die Bewegung in eine abklingende 
$e$-Funktion entartet.
\[ \boxed{x(t) \sim e^{(-\beta \pm \delta)t}} \]

\subsection{Kritische Dämpfung}
Der Punkt der kritischen Dämpfung (aperiodischer Grenzfall) beschreibt
die Grneze an der gerade noch keine Schwingung möglich ist. Die 
Abklingkonstante $\beta$ ist in diesem Fall gleich der Kreisfrequenz
$\omega$. 
\[ \boxed{b = b_{krit} = \sqrt{4 \cdot k \cdot m}} \]

\subsection{Gedämpfte Schwingung}
Im Falle der gedämpften Schwingung ist die Abklingkonstante $\beta$
kleiner als die Kreisfrequenz $\omega$. Somit schwingt das System
aber die Amplitude nimmt mit der Zeit $t$ ab\footnote{Es gibt je nach
physikalischem System verschiedene Arten der Dämpfung. Im Modul
MA+PHY2 wird stets mit der Dämpfung gerechnet welche die Amplitude
in einer $e$-Funktion abklingen lässt.}.
\[ \boxed{x(t) 
	= A \cdot e^{-\beta t} \cdot \cos(\omega_d \cdot t)
} \]
Die Dämpfung verändert die Kreisfrequenz $\omega_0$ des ungedämpften 
Systems zu $\omega_d$. Die Kreisfrequenz des gedämpften Systems ist
dabei immer kleiner als die des ungedämpften, also $\omega_d < \omega_0$.
\[ \boxed{\omega_d 
	= \sqrt{{\omega_0}^2 - \beta^2} 
	= \sqrt{\frac{k}{m} - \left({\frac{b}{2 \cdot m}}\right)^2}
} \]
Für grosse Güten kann die gedämpfte Kreisfrequnez mit der ungedämpften
approiximiert werden. Für kleine $Q$ (im Bereich $Q<10$) kann die
 gedämpfte Kreisfrequenz iterativ ermittelt werden mit
\[ \omega_d 
	=  \sqrt{{\omega_0}^2 - \frac{{\omega_d}^2}{4Q^2}}
\]
Für Güten grösser als $5$ reicht meist schon eine Iteration aus.

\subsubsection{Abklingkonstante}
\[ \boxed{\beta 
	= \frac{\ln\left(\frac{x(t_1)}{x(t_2)}\right)}{t_2 - t_1}
} \]

\subsubsection{Zerfallszeit}
Die Zerfallszeit $\tau$ einer gedämften Schwingung wird durch die 
Abklingkonstante $\beta$ bestimmt.
\[ \boxed{\tau 
	= \frac{1}{\beta}
} \]
Die Amplitude der Schwingung nimmt nach der $e$-Funktion ab. Diese kann
sowohl mit der Zerfallsszeit $\tau$ als auch mit der Abklingkonstante 
$\beta$ beschrieben werden.
\[ \boxed{A(t) 
	= A_0 \cdot e^{-\beta t} 
	= A_0 \cdot e^{-\frac{t}{\tau}}
} \]
Für die Praxis ergibt sich ein einfacher Zusmmenhang aus der $e$-Funktion,
denn diese definiert, dass pro verstrichenes $\tau$ die Amplitude um ca.
63\% gesunken ist. D.h. für $A(t=\tau) \approx A_0 \cdot 0.37$. Somit kann
das Abklingen nach $5\tau$ als abgeschlossen betrachtet werden.

\begin{figure}[h!]
	\centering
	\begin{tikzpicture}[domain=0:5]
		\draw[->] (0,0) -- (6,0) node[below] {$t$};
		\draw[->] (0,0) -- (0,2.5) node[left] {$A,\,E$};

		\draw[color=red, samples=200, thick] plot[id=a] function{2*exp(-1*x)};
		\draw[color=blue, samples=200, thick] plot[id=p] function{2*exp(-2*x)};

		\draw[color=red] (5,2.5) node[right] {$A(t)$};
		\draw[color=blue] (5,2) node[right] {$\langle E(t) \rangle$};
	
		\draw[] (0.1,2) -- (-0.1,2) node[left] {$A_0$};
		\draw[] (1,0.72) circle (1pt);
		\draw[dotted] (0,0.74) -- (1,0.74); 
		\draw[dotted] (1,0) -- (1,0.74);

		\draw[] (0.5,0.74) circle (1pt);
		\draw[dotted] (0.5,0) -- (0.5,0.74);

		\draw[] (0.1,0.74) -- (-0.1,0.74) node[left] {$\approx 0.37 \cdot A_0$};

		\draw[] (0.5,0.1) -- (0.5,-0.1) node[below] {$\frac{\tau}{2}$};
		\draw[] (1,0.1) -- (1,-0.1) node[below] {$\tau$};
		\draw[] (2,0.1) -- (2,-0.1) node[below] {$2\tau$};
		\draw[] (3,0.1) -- (3,-0.1) node[below] {$3\tau$};
		\draw[] (4,0.1) -- (4,-0.1) node[below] {$4\tau$};
		\draw[] (5,0.1) -- (5,-0.1) node[below] {$5\tau$};
	\end{tikzpicture}
	\caption{Abklingen einer Schwingung}
	\label{fig:abklingen}
\end{figure}
\subsubsection{Schwingungsenergie}
Die exakte Energie zur Zeit $t$ muss mittels der Bewegung gerechnet werden.
\[ \boxed{E(t) 
	= \frac{m}{2} \cdot \dot{x}^2(t) + \frac{k}{2} \cdot x^2(t)
} \]
Da diese Berechnung eher mühsam ist und der Verlauf der mittleren Energie 
oft mehr Aussagekraft hat, kann diese der exakten Berechnung vorgezogen 
werden.

Die Abnahme der Schwingungsenergie ist doppelt so schnell wie die Abnahme 
der Schwingung (siehe Abbildung \ref{fig:abklingen}). Somit kann diese 
analog zur Amplitude der gedämpften Schwingung formuliert werden als
\[ \boxed{ \langle E(t) \rangle 
	= E_0 \cdot e^{-2 \beta t} = E_0 \cdot e^{-\frac{2 t}{\tau}}
}\]
Wichtig ist hierbei noch zu bemerken, dass die Energie $E(t)$ eine
doppelt so hohe Frequenz hat wie die Bewegung $A(t)$. Dies ist so, da
die Energie pro Periode je zwei mal schwappt zwichen den Energiespeichern.

\begin{comment}
\begin{figure}[h!]
	\centering
	\begin{tikzpicture}[domain=0:6.28]
		\draw[->] (0,0) -- (7,0) node[below] {$t$};
		\draw[->] (0,0) -- (0,3) node[left] {$A,\,E$};

		\draw[color=red, samples=200] plot[id=f] function{sin(2*x)};
		\draw[color=blue, samples=200] plot[id=p] function{(1*sin(2*x))*(1*sin(2*x+pi/2))+0.5};

		\draw[color=red] (1,3) node[right] {$A(t)$};
		\draw[color=blue] (5,3) node[right] {$E(t)$};
	\end{tikzpicture}
\end{figure}
\end{comment}

\subsubsection{Güte}
Die Güte ist ein Mass für die Dämpfung. Diese beschreibt wie viele 
Schwingungen pro $\tau \cdot \pi$ stattfinden bis die Amplitude um 
63\% sinkt.
\[ \begin{array}{l} 
	\boxed{ Q	
		= 2\pi \frac{E(t)}{|\Delta E(t)|}
		= \pi \frac{1}{\beta T_d}
		= \pi \frac{\tau}{T_d}
		= \frac{\omega_d \tau}{2} 
	} \\
	\Delta E(t) \text{ ist der Energieverlust pro Periode } T_d
\end{array} \]
Je kleiner die Güte, desto mehr macht sich die Veränderung von $T_d$
bzw. $\omega_d$ bemerkbar. Für grosse Güten können die Werte als
gleich angenommen werden.
\[ \boxed{ Q \text{ gross} 
	\quad \Rightarrow \quad \omega_0 \approx  \omega_d
	\quad \Rightarrow \quad  T_0 \approx T_d
} \]
