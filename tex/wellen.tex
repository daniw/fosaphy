% coding:utf-8

%----------------------------------------
%FOSAPHY, a LaTeX-Code for a summary of basic physics
%Copyright (C) 2013, Daniel Winz, Ervin Mazlagic

%This program is free software; you can redistribute it and/or
%modify it under the terms of the GNU General Public License
%as published by the Free Software Foundation; either version 2
%of the License, or (at your option) any later version.

%This program is distributed in the hope that it will be useful,
%but WITHOUT ANY WARRANTY; without even the implied warranty of
%MERCHANTABILITY or FITNESS FOR A PARTICULAR PURPOSE.  See the
%GNU General Public License for more details.
%----------------------------------------

\chapter{Wellen}
\section{Allgemein}
\[ \boxed{f(x,t) = f^*(x \pm v \cdot t)} \]

\section{Seilwelle}
\begin{tabular}{@{}ll}
$F_S$:   & Seilkraft in $kg \frac{m}{s^2}$ \\
$\mu$:  & Längendichte in $\frac{kg}{m}$ \\
$v$:    & Geschwindigkeit
\end{tabular}
% Wellengleichung: 
% \[ \boxed{\frac{F_s}{\mu} \frac{\partial^2}{\partial x^2} f(x,t) 
% = \frac{\partial^2}{\partial t^2} f(x,t)} \]
% Gag: Jede Funktionen der Form $f(x,t) = f^*(x \pm v t)$ ist eine Lösung
\[ \boxed{v = \sqrt{\frac{F_s}{\mu}}} \]
\[ \boxed{t = \frac{s}{v}} \]
Anfangsbedingung (rechtslaufende Welle): 
\[ \boxed{f(x,0) = f^*(x) = A \cos(\frac{2 \pi}{\lambda} x + \phi)} \]
Allgemeine Lösung (rechtslaufende Welle): 
\[ \boxed{f(x,t) = f^*(x-vt) = A \cos (\frac{2 \pi}{\lambda}(x-vt)+\phi)} \]
\[ \boxed{f(x,t) = A \cos(kx-\omega t + \phi)} \]
\[ \boxed{\omega = \frac{2 \pi}{T}} \]
\[ \boxed{k = \frac{2 \pi}{\lambda} \quad \text{Wellenzahl}} \]
\[ \boxed{v = \frac{\lambda}{T} = \frac{\omega}{k}} \]
%\[ \boxed{} \]

\section{Druckwellen, Schallwellen}
Fluid: 
\[ \boxed{v = \sqrt{\frac{1}{\kappa \cdot \rho}}} \]
Kompressibilität: 
\[ \boxed{\kappa = -\frac{1}{V} \frac{d V}{d \rho}} \]
Festkörper: 
\[ \boxed{v = \sqrt{\frac{E}{\rho}}} \]
\begin{tabular}{ll}
\rowcolor{white} Medium & v \\\\
\rowcolor{lgray} Luft (20$^\circ$C) & 343 m/s \\
\rowcolor{white} Helium (20$^\circ$C) & 999 m/s \\
\rowcolor{lgray} Wasserstoff (20$^\circ$C) & 1330 m/s \\
\rowcolor{white} flüssiges Helium (4K) & 211 m/s \\
\rowcolor{lgray} Wasser (0$^\circ$C) & 1402 m/s \\
\rowcolor{white} Wasser (20$^\circ$C) & 1482 m/s \\
\rowcolor{lgray} Quecksilber (20$^\circ$C) & 1451 m/s \\
\rowcolor{white} Aluminium & 6420 m/s \\
\rowcolor{lgray} Blei & 1960 m/s \\
\rowcolor{white} Stahl & 5941 m/s \\
\end{tabular}

\section{Energiefluss}
\[ \boxed{P \propto A^2} \]
\subsection{Energiefluss der harmonischen Seilwelle}
\[ \boxed{P_{Seil} = \frac{1}{2} \mu v \cdot \omega^2 A^2 
= \frac{1}{2} \rho S v \omega^2 A^2} \]
\subsection{Energiefluss einer Schallwelle}
\[ \boxed{\hat{p} = \rho \omega v A} \qquad \text{Schalldruck} \]
\[ \boxed{P_{Schall} = \frac{1}{2} \rho S v \omega^2 A^2 
= \frac{1}{2} \frac{S}{\rho v} \hat{p}^2} \]

\section{Reflektion / Transmission}
\[ \boxed{P_{in} = P_r + P_t} \]
\[ \boxed{A_{in} + A_r = A_t} \]
\[ \boxed{\frac{P_2}{P_1} = \left(\frac{A_2}{A_1}\right)^2} \]

\section{Intensität einer Welle}
\subsection{Ebene Welle oder gerichteter Strahl}
\[ \boxed{I = \frac{P_{av}}{S}} \qquad S: \text{Querschnittsfläche}\]
\[ \boxed{I_{Seil} = \frac{1}{2} \rho v \omega^2 A^2} \]
\[ \boxed{I_{Schall} = \frac{1}{2} \frac{\hat{p}^2}{\rho v}} \]
\subsection{Zylinder Welle oder Kreisfläche}
\[ \boxed{I(r) = \frac{P_{av}}{S_{Zylinder}} = \frac{P_{av}}{L \cdot 2 \pi r} 
\propto \frac{1}{r}} \]
\subsection{Kugelwelle}
\[ \boxed{I(r) = \frac{P_{av}}{S_{Kugelschale}} = \frac{P_{av}}{4 \pi r^2} 
\propto \frac{1}{r^2}} \]

\section{Schallpegel}
\[ \boxed{L = 10 \cdot \log\left(\frac{I}{I_0}\right)} 
\qquad I_0 = 10^{-12} \frac{W}{m^2} \text{(Hörschwelle)} \]
\[ \boxed{I = I_0 \cdot 10^{\frac{L_2 - L_1}{10}}} \]

\section{Dopplereffekt}
Bewegter Sender: 
\[ \boxed{f' = \frac{v}{\lambda \pm v_S T} = \frac{v}{v \pm v_S}f} \]
Bewegter Empfänger: 
\[ \boxed{f' = \frac{v \pm v_E}{\lambda} = \frac{v \pm v_E}{v}f} \]
Kombiniert: 
\[ \boxed{f' = \frac{v \pm v_E}{v \pm v_S}f} \qquad \substack{\text{
$+$, wenn e sich auf Quelle zu bewegt. }\\\\\text{
$+$, wenn s sich vom Empfänger weg bewegt. }} \]

\subsection{Machscher Kegel}
\[ \boxed{\sin(\theta) = \frac{v}{v_s} = \frac{1}{Ma}} \]
\begin{tabular}{l@{}l}
$\theta$:   & Halber Öffnunswinkel des Kegels \\
$v$:        & Schallgeschwindigkeit \\
$v_s$:      & Geschwindigkeit des Senders
\end{tabular}

\subsection{Superposition}
Konstruktiv: 
\[ \boxed{A = A_1 + A_2} \]
\[ \boxed{P = \left(\sqrt{P_1} + \sqrt{P_2}\right)^2} \]
Destruktiv: 
\[ \boxed{A = A_1 - A_2} \]
\[ \boxed{P = \left(\sqrt{P_1} - \sqrt{P_2}\right)^2} \]

\section{Schwebung}
Resultierende Welle
\[ \boxed{2 \pi \cdot f_{av} = \frac{\omega_1 + \omega_2}{2}} \]
Umhüllkurve
\[ \boxed{2 \pi \cdot f_B = |\omega_1 - \omega_2|} \]

\section{Stehende Welle}
\[ \boxed{f_{ges}(x,t) = 2 \cdot A \cdot \sin(k x) \cdot \sin(\omega t)} \]
$A$: Amplitude der eingehenden Sinusfunktion \\

\subsection{Harmonische}
\begin{tabular}{@{}lllll}
\rowcolor{white} $n$ & $\lambda$       & \#Bäuche & \#Knoten & Mitte \\
\rowcolor{lgray} $1$ & $\frac{2 L}{1}$ & $1$      & $0$      & Bauch \\
\rowcolor{white} $2$ & $\frac{2 L}{2}$ & $2$      & $1$      & Knoten \\
\rowcolor{lgray} $3$ & $\frac{2 L}{3}$ & $3$      & $2$      & Bauch \\
\rowcolor{white} $4$ & $\frac{2 L}{4}$ & $4$      & $3$      & Knoten \\
\rowcolor{lgray} $5$ & $\frac{2 L}{5}$ & $5$      & $4$      & Bauch \\
\end{tabular}
% \subsection{n-te Harmonische}
\[ \lambda_n = \frac{2 L}{n} \qquad f_n = \frac{v}{\lambda_n} 
= n \frac{v}{2 L} = n \cdot f_1 \qquad 
\text{$f_1$: Grundschwingung oder 1. Harmonische}\]
\#Bäuche: $n$ \\
\#Knoten: $n - 1$ \\
Mitte: \\
$n$ gerade: Knoten \\
$n$ ungerade: Bauch
\[ f_n(x, t) 
= A \sin\left(\frac{2 \pi}{\lambda_n} x\right) \cdot \sin(2 \pi f_u t) \]

\subsection{Spezialfall Orgelpfeife}
\begin{tabular}{@{}llll}
\rowcolor{white} $n$ & $\lambda_n$     & \#Bäuche & \#Knoten  \\
\rowcolor{lgray} $1$ & $\frac{4 L}{1}$ & $1$      & $0$       \\
\rowcolor{white} $3$ & $\frac{4 L}{3}$ & $2$      & $1$       \\
\rowcolor{lgray} $5$ & $\frac{4 L}{5}$ & $3$      & $2$       \\
\rowcolor{white} $7$ & $\frac{4 L}{7}$ & $4$      & $3$       \\
\rowcolor{lgray} $9$ & $\frac{4 L}{9}$ & $5$      & $4$       \\
\end{tabular}

\subsection{Bäuche und Knoten finden}
\[ \begin{array}{@{}llll}
\text{Bäuche:} \\
\sin(k x + \phi) = 1 
    & \rightarrow 
    & k x + \phi = n \cdot \pi + \frac{\pi}{2} 
    & , n \in \mathbb{Z} \\
\cos(k x + \phi) = 1 
    & \rightarrow 
    & k x + \phi = n \cdot \pi                 
    & , n \in \mathbb{Z} \\\\
\text{Knoten:} \\
\sin(k x + \phi) = 0 
    & \rightarrow 
    & k x + \phi = n \cdot \pi                 
    & , n \in \mathbb{Z} \\
\cos(k x + \phi) = 0 
    & \rightarrow 
    & k x + \phi = n \cdot \pi + \frac{\pi}{2} 
    & , n \in \mathbb{Z} \\
\end{array} \]

\section{Wasserwelle}
Tiefwasserwelle
\[ \boxed{v = \sqrt{\frac{g}{k}} = \sqrt{\frac{g \cdot \lambda}{2 \pi}}} \]
Seichtwasser
\[ \boxed{v = \sqrt{g \cdot y}} \]
y: Wassertiefe
