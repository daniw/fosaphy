% coding:utf-8

%----------------------------------------
%FOSAPHY, a LaTeX-Code for a summary of basic physics
%Copyright (C) 2013, Mario Felder

%This program is free software; you can redistribute it and/or
%modify it under the terms of the GNU General Public License
%as published by the Free Software Foundation; either version 2
%of the License, or (at your option) any later version.

%This program is distributed in the hope that it will be useful,
%but WITHOUT ANY WARRANTY; without even the implied warranty of
%MERCHANTABILITY or FITNESS FOR A PARTICULAR PURPOSE.  See the
%GNU General Public License for more details.
%----------------------------------------

\chapter{Bewegung}

\section{Gerade Bewegung}

\[v=\lim\limits_{\Delta t \rightarrow 0}{\frac{\Delta x}{\Delta t}}=\frac{\mathrm{d}x}{\mathrm{d}t}=\dot{x}\]
\[a=\lim\limits_{\Delta t \rightarrow 0}{\frac{\Delta v}{\Delta t}}=\frac{\mathrm{d}v}{\mathrm{d}t}=\dot{v}=\ddot{x}\]
\[\Delta x=\lim\limits_{\Delta t_i \rightarrow 0}{\sum^{n}_{i=1}v_i\cdot\Delta t}=\int_{t_A}^{t_B}v\mathrm{d}t\]

\subsection{Spezialfall: konstante Beschleunigung a}

\[a(t)=a=\mathrm{konstant}\]
\[v(t)=v_0+a\cdot t\]
\[x(t)=x_0+v_0\cdot t+\frac{1}{2}a^{2}\]
\[\Delta x=x-x_0=v_0\cdot t+\frac{1}{2}\cdot a\cdot t^2\]
\[v^2=v_{0}^{2}+2a\Delta x\]
\[a=\frac{v(t)^2-v_0^2}{2\cdot\Delta s}\]

\section{Bewegung im Raum}
Postition, Geschwindigkeit und Beschleunigung sind Vektoren.\\

\begin{footnotesize}
\boxed{
\begin{tabular}{l}
		$\vec{\Delta r}=\vec{r_2}-\vec{r_1}=(x_2-x_1,y_2-y_1,z_2-z_1)$\\ \\
		$v\rightarrow\vec{v}=\lim\limits_{\Delta t \rightarrow 0}{\frac{\vec{\Delta r}}{\Delta t}}
			=\frac{\mathrm{d}\vec{r}}{\mathrm{d}t}
			=(\frac{\mathrm{d}x}{\mathrm{d}t},\frac{\mathrm{d}y}{\mathrm{d}t},\frac{\mathrm{d}z}{\mathrm{d}t})$\\ \\
		$a\rightarrow\vec{a}=\lim\limits_{\Delta t \rightarrow 0}{\frac{\vec{\Delta v}}{\Delta t}}
			=\frac{\vec{\mathrm{d}v}}{\mathrm{d}t}
			=\frac{\mathrm{d^2}\vec{r}}{\mathrm{d}t^2}$
\end{tabular}
}
\end{footnotesize}
\newline


\subsection{Bahnkurve}

Die Geschwindigkeit liegt immer \textbf{tangential} an der Bahnkurve. \newline
\newline
Die Beschleunigung zeigt immer nach \textbf{innen}.

\subsection{Kreisbewegung}
Bei einer \textbf{gleichförmigen} Kreisbewegung ($v=\mathrm{konst.}$) gilt:
\[
	a_{ZP}=a_{radial}=\frac{v^2}{r}=\omega^2 \cdot r
\]
\newline
\newline
Bei einer \textbf{ungleichförmigen} Kreisbewegung gilt:
\[
	a_{radial}=\frac{v^2}{r}=\omega^2 \cdot r \\
	a_{tangential}=\frac{\mathrm{d}\left|v\right|}{\mathrm{d}t}
\]

\section{Schiefer Wurf}

x- und y-Bewegung sind unabhängig:
\newline
\begin{footnotesize}
\boxed{
\begin{tabular}{ll}
		\textbf{horizontal:}						&	\textbf{vertikal:}\\
		$a_x=0$													&	$a_y=-g$ \\
		$v_x=v_0*\cos\alpha_0$					& $v_y=v_0*\sin\alpha_0 - g \cdot t$ \\
		$x=(v_0*\cos\alpha_0) \cdot t$	& $y=(v_0*\sin\alpha_0) \cdot t - \frac{1}{2}g \cdot t^2$ \\
																		&	$v_y^2=v_{0y}^2-2gy$
\end{tabular}
}
\end{footnotesize}
\newline
\newline
Wurfdauer:
\[
	\boxed{t_R=\frac{2 \cdot v_{0y}}{g}=\frac{2v_0 \cdot\sin\theta}{g}}
\]
\newline
Wurfweite:
\[
	\boxed{R=x(t_r)=\frac{v_0^2\sin2\theta}{g}}
\]
\newline

\subsection{$x(t),y(t)\leftrightarrow y(x)$}
\[
	\boxed{
		y(x)=\tan\theta_0 \cdot x-\frac{g}{2(v_0\cos\theta_0)^2}\cdot x^2
	}
\]
\newline
\begin{figure}[htbp]
\centering
\begin{gnuplot}[scale=0.62]
	set terminal epslatex color
	set grid
  set xrange [0:100]
	set yrange [-80:35]
	set xtics 20
	set ytics 20
	set xlabel '$x$ [m]'
	set ylabel '$y$ [m]'
	set xzeroaxis
	
  # define the function
  f(x,theta)=tan(pi/180*theta)*x-9.81/(2*(25*cos(pi/180*theta))**2)*x**2

  plot f(x,70) ti "$\\theta=70$",f(x,53.1) ti "$\\theta=53.1$",f(x,45) ti "$\\theta=45$",f(x,36.9) ti "$\\theta=36.9$",f(x,20) ti "$\\theta=20$"
\end{gnuplot}
\end{figure}

\subsection{schräge Zerlegung}
Die Komponentenzerlegung eines Vektors ist beliebig. Manchmal ist eine schräge Zerlegung besser als eine Senkrechte,
beispielsweise in die $\vec{v_0}$ und $\vec{g}$ Richtung:
\newline
\[
	s=v_0 \cdot t \\
	y=\frac{1}{2}g \cdot t^2
\]
\begin{figure}[htbp]
\centering
\begin{gnuplot}[scale=0.75]
	set terminal epslatex color
	set size ratio -1
	set grid
  set xrange [0:55]
	set yrange [-5:15]
	set xtics 5
	set ytics 5
	set format ""
	unset key

  # define the function
  f(x,theta)=tan(pi/180*theta)*x-9.81/(2*(25*cos(pi/180*theta))**2)*x**2

	set style line 2 lc rgb 'black' pt 7   # circle
	
	set label 1 "$s=v_0 \\cdot t$" at 25,10 rotate by 20
	set arrow from 0,0 to 35,12.739
	
	set label 2 "$\\vec{v_0}$" at 10,5
	set arrow from 0,0 to 15,tan(pi/180*20)*15 linecolor rgb 'green' linewidth 3
	
	set label 3 "$y=\\frac{1}{2}g \\cdot t^2$" at 37,11 rotate right
	set arrow from 35,12.739 to 35,f(35,20)
	
	set label 4 "$\\vec{v_0}$" at 45,7
	set arrow from 35,f(35,20) to 50,tan(pi/180*20)*15+f(35,20) linecolor rgb 'green' linewidth 3
	
	set object circle at 35,f(35,20) size 0.2
  plot f(x,20) ti ""
	\end{gnuplot}
\end{figure}

\subsection{Kraft}
\[ F = \frac{dp}{dt} = m \dot a \]

\subsection{Impuls}
\[ \vec{p} = m \cdot \vec{v} \]

\subsection{Kraftstoss}
\[ \vec{J} = \int_{t_1}^{t2} (F(t)) ~ dt = \vec{p_2} - \vec{p_1} \]

\subsection{Kinetische Energie}
\[ E_{kin} = \frac{1}{2} \cdot m \cdot v^2 = \frac{p^2}{2 \cdot m} \]

\subsection{Leistung}
\[ P = \frac{dE}{dt} = \frac{F \cdot ds}{dt} = F \cdot v \]

