% coding:utf-8

%----------------------------------------
%FOSAPHY, a LaTeX-Code for a summary of basic physics
%Copyright (C) 2013, Daniel Winz, Ervin Mazlagic

%This program is free software; you can redistribute it and/or
%modify it under the terms of the GNU General Public License
%as published by the Free Software Foundation; either version 2
%of the License, or (at your option) any later version.

%This program is distributed in the hope that it will be useful,
%but WITHOUT ANY WARRANTY; without even the implied warranty of
%MERCHANTABILITY or FITNESS FOR A PARTICULAR PURPOSE.  See the
%GNU General Public License for more details.
%----------------------------------------

\chapter{Wärme}

\newpage
\section{Begriffe}
\begin{table}[h!]
\rowcolors{1}{lgray}{white}
\renewcommand{\arraystretch}{1.5}
\begin{tabular}{p{0.1\textwidth} l c}
$n$		
	& Stoffmenge 		
	& $[mol]$ \\
$N$		
	& Anzahl Teilchen 	
	& - \\
$m$		
	& Masse 		
	& $[kg]$ \\
$m_{mol}$	
	& Molare Masse (siehe Periodensystem, S. \pageref{fig:persys})	
	& $\left[\frac{g}{mol}\right]$ \\
$T$		
	& Tempteratur in Kelvin  ($T = T_{^\circ C} + 273.15 ^\circ C$) 
	& $[K]$ \\
$N_A$	
	& Avogadrozahl ($6.022~141~3 \cdot 10^{23} \frac{1}{mol}$) 
	& - \\
$k_B$		
	& Boltzmannkonstante ($1.380~648~8 \cdot 10^{-23}$)
	& $\left[\frac{J}{K}\right]$ \\ 
$R$	
	& Gaskonstante ($8.314'462'1$)
	& $\left[\frac{J}{mol \cdot K}\right]$
\end{tabular}
\end{table}

\newpage

\begin{comment}
\section{Begriffe}
\[ \begin{array}{@{}ll} 
n:      & \text{Stoffmenge $[mol]$} \\
N:      & \text{Anzahl Teilchen} \\
m:      & \text{Masse} \\
m_{mol}:& \text{Molare Masse 
          (Siehe Periodensystem auf Seite \pageref{fig:persys})} 
          \left\lbrack\frac{g}{mol}\right\rbrack \\
T:      & \text{Tempteratur in $K$ } 
            (T = T_{^\circ C} + 273.15 ^\circ C) \\
N_A:    & \text{Avogadrozahl }
            (6.022~141~3 \cdot 10^{23} \frac{1}{mol}) \\
k_B:    & \text{Boltzmannkonstante } 
            (1.380~648~8 \cdot 10^{-23} \frac{J}{K} ) \\
R:      & \text{Gaskonstante } 
            (8.314~462~1 \frac{J}{mol \cdot K}) \\
\end{array} \]
\end{comment}

\section{Berechnungen zu molaren Grössen}
\[ \boxed{n = \frac{N}{N_A} = \frac{m}{m_{mol}}} \]
Molare Masse von Gasgemischen: 
\[ \boxed{m_{mol_{gem}} = m_{mol_{a}} \cdot \text{Anteil}_a 
+ m_{mol_{b}} \cdot \text{Anteil}_b + \dots} \]
Molare Masse von Molekülen: 
\[ \boxed{m_{kül} = x_1 \cdot m_{mol_a} + x_2 \cdot m_{mol_b} + \dots } 
\qquad x_1, x_2: \text{Anzahl Atome} \]

\section{Gasgleichung}
\[ \boxed{\frac{p \cdot V}{T \cdot n} = const} \]
\[ \boxed{p \cdot V = N \cdot k_B \cdot T 
= n \cdot R \cdot T 
= \frac{N}{N_A} \cdot R \cdot T 
= \frac{m}{m_{mol}} \cdot R \cdot T } \]

\section{Spezifische Wärmekapazität}
\[ \boxed{Q = m \cdot c \cdot \Delta T} \]
\[ \boxed{dQ = m \cdot c \cdot d T} \]
\[ \boxed{c_{(m)} = \frac{1}{m} \frac{dQ}{d T} \qquad \text{c pro Masse}} \]
\[ \boxed{c_{(n)} = \frac{1}{n} \frac{dQ}{d T} \qquad \text{c pro Mol}} \]

\subsection{Spezifische Wärmekapazität von Gasen}
konstantes Gasvolumen
\[ \boxed{C_{(n)V} = C_{V} = \frac{\#FG}{2}R} \]
\begin{tabular}{lcc}
                                      & \#FG     & $C_V$ \\
Ideale Gase: ($Ar$, $Ne$, $He$, ...)  & $3$      & $\nicefrac{3}{2} R$ \\
zwei-atomige: ($O_2$, $N_2$, CO, ...) & $\sim 5$ & $\nicefrac{5}{2} R$ 
\end{tabular}

\subsection{Spezifische Wärme $C_p$ des idealen Gas}
\[ \boxed{C_p = C_v + R} \]
\[ \boxed{\gamma \equiv \frac{C_p}{C_v} = \frac{C_v + R}{C_v}} \]
\[ \gamma = 1.67 \qquad \text{1 atomiges Gas} \]
\[ \gamma = 1.40 \qquad \text{2 atomiges Gas} \]

\section{Volumenarbeit eines Gases}
\[ \boxed{W = -\int_{V_1}^{V_2} p \cdot dV = - p \cdot (V_2 - V_1)} \]

\section{Innere Energie des idealen Gas}
\[ \boxed{U(T) = n \cdot \frac{\#FG}{2} \cdot R \cdot T = n \cdot C_v \cdot T} \]

\section{Zustandsänderungen des idealen Gas}
\[  \begin{array}{ll} 
\Delta p: & \text{Druckänderung} \\
\Delta Q: & \text{Wärmeänderung} \\
\Delta T: & \text{Temperaturänderung} \\
\Delta U: & \text{Änderung der inneren Energie} \\
\Delta V: & \text{Volumenänderung} \\
\Delta W: & \text{Arbeit} \\
\end{array} \]

\subsubsection{isochorer Prozess}
Keine Volumenänderung
\[ \boxed{\Delta V = 0} \qquad \boxed{\Delta W = 0} \]
\[ \boxed{\Delta Q = n \cdot C_v \cdot (T_2 - T_1) = \Delta U} \]

\subsubsection{isobarer Prozess}
Keine Druckänderung
\[ \boxed{\Delta p = 0} \]
\[ \boxed{\Delta U = n \cdot C_v \cdot (T_2 - T_1)} \]
\[ \boxed{\Delta W = - p \cdot (V_2 - V_1) = - n \cdot R \cdot (T_2 - T_1)} \]
\[ \boxed{\Delta Q = n \cdot C_p \cdot (T_2 - T_1)} \]

\subsubsection{isothermer Prozess}
Keine Temperaturänderung
\[ \boxed{\Delta T = 0} \qquad \boxed{\Delta U = 0} \]
\[ \boxed{\Delta W = \int_{V_1}^{V_2} - p dV = - n \cdot R \cdot T \cdot 
\ln\left(\frac{V_2}{V_1}\right) = - \Delta Q} \]

\subsubsection{adiabatischer Prozess}
kein Wärmeaustausch
\[ \boxed{\Delta Q = 0} \]
\[ \boxed{\Delta U = n \cdot C_v \cdot (T_2 - T_1) = \Delta W
= \frac{p_2 \cdot V_2 - p_1 \cdot V_1}{\gamma - 1}} \]
\[ \boxed{T_1 \cdot {V_1}^{(\gamma - 1)} = T_2 \cdot {V_2}^{(\gamma - 1)}} \]
\[ \boxed{p_1 \cdot {V_1}^{\gamma} = p_2 \cdot {V_2}^{\gamma}} \]
\[ \boxed{T_1 \cdot {p_1}^{\left(\frac{1 - \gamma}{\gamma}\right)} 
= T_2 \cdot {p_2}^{\left(\frac{1 - \gamma}{\gamma}\right)}} \]

\section{Druck}
\[ \boxed{p = \frac{F}{A} 
= m_{mol} \cdot {\langle v_x\rangle}^2 \cdot \frac{N}{V}} \]

\section{Barometrische Höhenformel}
Konstante Temperatur
\[ \boxed{p(y) 
= p_0 \cdot e^{- \frac{m_{mol} \cdot g}{R \cdot T} \cdot y} 
= p_0 \cdot e^{- \frac{y}{H}}}\]
\[ \boxed{H = \frac{R \cdot T}{m_{mol} \cdot g} 
\qquad \text{Charakteristische Höhe}} \]
Veränderliche Temperatur
\[ \boxed{p(y) = p_0 \cdot \left(1 - \frac{b}{a} \cdot y\right)^
{\frac{m_{mol} \cdot g}{R \cdot b}} 
= p_0 \cdot \left(\frac{T}{T_0}\right)^
{\frac{m_{mol} \cdot g}{R \cdot b}}} \]
$a = 288.15 K$ \\
$b = 6.5 \frac{K}{km}$ \\

\section{Kinetische Gastheorie}
\[ \boxed{k_B = \frac{R}{N_A}} \]
\[ \boxed{p = \frac{N}{V} \cdot \frac{1}{3} \cdot  m_{kül} \cdot \langle v^2\rangle  
= \frac{N}{V} \cdot \frac{2}{3} \cdot \langle E_{kin, kül}\rangle  
= \frac{N}{V} \cdot  k_B \cdot T} \]
mikroskopisch: 
\[ \boxed{\langle E_{kin, kül}\rangle  
= \frac{1}{2} \cdot m_{kül} \cdot  \langle v^2\rangle 
= \frac{3}{2} \cdot k_B \cdot T} \]
makroskopisch: 
\[ \boxed{E_{Gas} = n \cdot \frac{3}{2} \cdot R \cdot T} \]

\section{Mittlere freie Weglänge}
\[ \boxed{t_{mean} = \frac{d t}{d N_{Stoss}} 
= \frac{V}{4 \cdot \pi \cdot \sqrt{2} \cdot r^2 \cdot v \cdot N}} \]
\[ \boxed{\Lambda = v \cdot t_{mean} 
= \frac{V}{4 \cdot \pi \cdot \sqrt{2} \cdot r^2 \cdot N} 
= \frac{k_B \cdot T}{4 \cdot \pi \cdot \sqrt{2} \cdot r^2 \cdot p}} \]
\[ \boxed{v_{rms} = \sqrt{\langle v^2\rangle} 
= \sqrt{\frac{3 \cdot k_B \cdot T}{m_{kül}}} 
= \sqrt{\frac{3 \cdot R \cdot T}{m_{mol}}}} \]

\section{Maxwell-Boltzmann Verteilung}
\[ \boxed{f(v) = 4 \pi \left(\frac{m_{mol}}{2 \pi \cdot R \cdot T}\right)
^{\frac{3}{2}} \cdot v^2 \cdot e^{-\left(\frac{m_{mol} \cdot v^2}
{2 \cdot R \cdot T}\right)}} \]
\[ \boxed{f(v) = 4 \pi \left(\frac{m_{kül}}{2 \pi \cdot k_B \cdot T}\right)
^{\frac{3}{2}} \cdot v^2 \cdot e^{-\left(\frac{m_{kül} \cdot v^2}
{2 \cdot k_B \cdot T}\right)}} \]
Wahrscheinlichkeit:
\[ \boxed{P(v_1, v_2) = \int\limits_{v_1}^{v_2} f(v) ~ dv} \]

\section{Charakteristische Geschwindigkeiten}
Wahrscheinlichste Geschwindigkeit
\[ \boxed{v_w = \sqrt{\frac{2 \cdot k_B \cdot T}{m_{kül}}} 
= \sqrt{\frac{2 \cdot R \cdot T}{m_{mol}}}} \]
Durchschnittsgeschwindigkeit
\[ \boxed{v_{av} = \langle v \rangle 
= \sqrt{\frac{8 \cdot k_B \cdot T}{\pi \cdot m_{kül}}} 
= \sqrt{\frac{8 \cdot R \cdot T}{\pi \cdot m_{mol}}}} \]
Root Mean Square Geschwindigkeit
\[ \boxed{v_{rms} = \sqrt{\langle v^2 \rangle} 
= \sqrt{\frac{3 \cdot k_B \cdot T}{m_{kül}}} 
= \sqrt{\frac{3 \cdot R \cdot T}{m_{mol}}}} \]
