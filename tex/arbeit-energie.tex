\chapter{Arbeit und Energie}

Energie ist ein zentraler Begriff der Pyhsik und eine wichtige 
Erhaltungsgrösse. Die Arbeit beschreibt im Grunde genommen das selbe wie 
Energie, jedoch wird es sprachlich anders eingesetzt. Als Arbeit 
bezeichent man in der Regel jene Energie, welche mittels eines Weges 
formuliert werden kann und von einem Körper auf einen anderen übertragen 
wird. Die Energie wird oft vereinfacht beschrieben als die Fähigkeit
Arbeit zu verrichten.

\newpage
\section{Arbeit}
Arbeit ist ein Ausdruck der verwendet wird um eine mechanische 
Energieübertrgung zu beschreiben. Diese beschreibt das Produkt aus
Kraft und Weg, wobei diese vektoriell sind.
\[ \boxed{ W = \vec{F} \cdot \vec{s} 
	= ||\vec{F}|| \cdot ||\vec{s}|| \cdot 
	\sphericalangle \left( \vec{F}, \vec{s} \right) } 
	\qquad \left[ J = N \cdot m = \frac{kg \cdot m^2}{s^2} \right] \]

\subsection{Goldene Regel der Mechanik}
Die goldene Regel der Mechanik besagt, dass man Arbeit als konstantes 
Produkt aus Kraft und Weg betrachten kann. Dies ist die Definition des
Energieerhaltungssatzes für den Bereich der Mechanik (z.B. einfache
Maschinen wie Flaschenzüge). 
\[ \boxed{W = \vec{F} \cdot \vec{s}} \] 
Dies bedeutet, man kann für die gleiche Arbeit jeweils einen verschieden
langen Weg nehmen. So kann die hierzu nötige Kraft variiert werden. 
Wichtig bei der Anwendung dieser goldenen Regel ist, dass der Weg als
auch die Kraft als vekorielle Grössen verstanden werden. 

\section{Energie}
Die Energie ist eine fundamentale Grösse aller Teilbereiche der Physik.
Diese kann in verschiedenen Formen auftreten. In der Mechanik werden vor
allem die potentielle und kinetische Energie behandelt.

\subsection{Potentielle Energie}
Die potentielle Energie ist definiert als jene Energie, welche sich aus
einem Höhenunterschied (Potentialdifferenz) einer Masse zu einer 
Referenzhöhe ergibt. 
\[ \boxed{ E_{Pot} 
	= m \cdot \vec{g} \cdot (h - h_{Ref})
	= m \cdot \vec{g} \cdot \Delta h 
} \]

\subsection{Kinetische Energie}
Die kinetische Energie ist proportional zur Masse des sich bewegenden Körpers
und quadratisch zu deren Geschwindigkeit. Wichtig ist hierbei auch die 
Betrachtung der Bewegung des Körpers, vor allem zu welcher Referez die 
Geschwindigkeit gilt (Relativgeschwindugkeit).
\[ \boxed{ E_{Kin_{Translation}} = \frac{m \cdot \vec{v}^2}{2} } \]
Die kinetische Energie kann aber auch eine Rotation beschreiben.
\[ \boxed{ E_{Kin_{Rotation}} = \frac{I \cdot \omega^2}{2} } \]

\subsection{Federenergie}
Die Federengergie ist definiert als das Produkt aus Federkonstante $k$ und 
dem Integral der Dehnung bzw. des Weges $s$ 
(siehe Kapitel \ref{sec:feder-energie}).
\[ \boxed{E_{Feder} 
	= k \cdot \int_{\vec{s}_a}^{\vec{s}_b} \vec{s} \cdot d\vec{s} 
	= \frac{k \cdot \vec{s}^2}{2}} \]

\subsection{Reibungsenergie}
Analog zur potentiellen Energie ist die Reibungsenerige durch eine Kraft
$\vec{F}_N \cdot \mu$ und einen Weg $\vec{s}$ definiert. 
$\vec{F}_N$ ist dabei die Kraft, welche senkrecht zur reibenden Unterlage 
wirkt, also $\vec{F}_N \bot \vec{s}$.
\[ \boxed{E_{Reibung} = \vec{F}_N \cdot \mu \cdot \vec{s}} \]
Diese kann auch allgemein formuliert werden mittels des Integrals über
den zurückgelegten Weg.
\[ \boxed{
	E_{Reibung} = \mu \cdot \int \left( 
		\vec{F}_N \cdot \vec{s} \right) ds} 
\]

\section{Leistung}\label{sec:leistung}
Die Leistung $P$ einer Kraft ist die Rate, mit der sie Arbeit zuführt,
also die Arbeit pro Zeit.
\[ \boxed{P = \frac{dW}{dt} } \]
Mit der Überlegung, dass die Arbeit $W$ das Produkt aus Kraft und Weg
ist, kann die Leistung auch als $\vec{F} \cdot \vec{s}$ beschrieben
werden. Mit dieser Formulierung lässt sich die Arbeit auch als Kraft mal
Geschwindigkeit aufstellen.
\[ \boxed{
	P 
		= \frac{dW}{dt} 
		= \frac{\vec{F} \cdot d\vec{s}}{dt}
		= \vec{F} \cdot \frac{d\vec{s}}{dt} 
		= \vec{F} \cdot \vec{v}
} \]
Hierbei gilt es wieder zu beachten, dass dies vektorielle Grössen sind
und der jeweilige Winkel dazwischen zu berücksichtigen ist.

\subsection{Mittlere Leistung}
Mit der Definition aus den Kapitel \ref{sec:leistung} kann nun auch eine
mittlere Leistung formuliert werden mittels der Integration zwischen
zwei Zeitpunkten $t_1$ und $t_2$.
\[ \boxed{
	P_{average} 
		= \frac{\displaystyle\int_{t_1}^{t_2} P\cdot dt}
			{t_2 - t_1}
		= \frac{\Delta W}{\Delta t} 
} \]
