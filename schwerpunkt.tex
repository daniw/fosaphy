\chapter{Schwerpunkt}

\section{Definition}
Der Schwerpunkt eines Systems bezeichnet eine vektorielle Grösse,
welche das Zentrum der Masse auszeichnet. Dieser wird bestimmt
durch den Quotienten aus der Summe aller Massen zu einem Abstand
und der Gesamtmasse.
\[ \boxed{ cm =  
	\begin{pmatrix} 
		x_{cm} \\ 
		\\
		y_{cm} \\
		\\
		z_{cm}
	\end{pmatrix} 
	=
	\begin{pmatrix}
		\frac{m_1 x_1 + m_2 x_2 + \dots + m_n x_n}
			{m_1 + m_2 + \dots + m_n} \\
		\\
		\frac{m_1 y_1 + m_2 y_2 + \dots + m_n y_n}
			{m_1 + m_2 + \dots + m_n} \\
		\\
		\frac{m_1 z_1 + m_2 z_2 + \dots + m_n z_n}
			{m_1 + m_2 + \dots + m_n}
	\end{pmatrix} }
\]
Der Ursprung des Koordinatesnsystems spielt bei der Ermittlung des
Schwerpunktes keine Rolle, denn es ist ein geometrischer Ort des 
betrachteten Objektes. Eine Verscheibung des Koordinatesnsystems
verändert diesen nicht.

\section{Bewegung des Schwerpunktes}

\subsection{Geschwindigkeit}
Wendet man das Newton'sche Bewegungsgesetz auf ein System von
Massen an, so gilt die Bewegung lediglich für den Schwerpunkt,
d.h. alle anderen Massenpnkte welche nicht den Schwerpunkt abbilden,
habe eine andere Bewegung bzw. können eine andere Bewegung haben.
\[ \boxed{ \vec{v}_{cm} = 
	\begin{pmatrix} 
		\frac{m_1 \vec{v_x}_1 
			+ m_2 \vec{v_x}_2 
			+ \dots m_n \vec{v_x}_n}
			{m_1 + m_2 + \dots + m_n} \\
		\\
		\frac{m_1 \vec{v_y}_1 
			+ m_2 \vec{v_y}_2 
			+ \dots m_n \vec{v_y}_n}
			{m_1 + m_2 + \dots + m_n} \\
		\\
		\frac{m_1 \vec{v_z}_1 
			+ m_2 \vec{v_z}_2 
			+ \dots m_n \vec{v_z}_n}
			{m_1 + m_2 + \dots + m_n}
	\end{pmatrix} }
\]
Um ein praktisches Beispiel zu geben kann man sich einen Ball vorstellen.
Der Ball (Schwerpunkt) bewegt sich mit $\vec{v} = m \cdot \vec{a}$.
Die einzelnen Massenpunkte (z.B. Ventil des Balls) muss dieser Bewegung
des Schwerpunktes nicht gleich sein, denn es kann um den Ball herum
rotieren, d.h. eine andere Geschwidigkeit haben und auf einer anderen
Bahn durch den Raum fliegen.

\subsection{Impuls}
Betrachtet man ein System aus Massen, d.h. ein Objekt aus 
Massenpunkten\footnote{Ein Objekt aus Massepunkten ist alles ausser 
einer rein theoretischen Punktmasse ohne Volumen, d.h. ein reales Objekt
wie man es aus dem Alltag kennt.}, so kann
auf dieses System mit Hilfe des Schwerpunktes ein Gesamtimpuls
formuliert werden.
\[ \boxed{
	\left( \sum_{i=1}^n m_i \right) \cdot \vec{v}_{cm} 
		= m_1 \vec{v}_1 + m_2 \vec{v}_2 + \dots m_n \vec{v}_n
		=\vec{p} }
\]

\subsection{Beschleunigung}
Ausgehend von der Geschwindigkeit kann die Beschleunigung des 
Massensystems bestimmt werden. Dies erreicht man durch die 
Ableitung des Gesamtimpulses nach der Zeit $dt$.
\[ \boxed{
	\left( \sum_{i=1}^n m_i \right) \cdot \vec{a}_{cm} 
		= m_1 \vec{a}_1 + \dots m_n \vec{a}_n
		= \frac{d \vec{P}}{dt} 
		= \vec{F}_{Res}
		= \sum_{i=1}^n \vec{F}_{extern} }
\]
Dies beschreibt das Newton'sche Gesetzt welches besagt, dass die 
Summe der anliegenden (externen) Kräfte das Produkt aus Masse und
Beschleunigung bildet. Diese vereinfachte Formulierung bezieht sich
wiederum auf den Schwerpunkt und deren Bewegung.


