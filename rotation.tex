% coding:utf-8

%----------------------------------------
%FOSAPHY, a LaTeX-Code for a summary of basic physics
%Copyright (C) 2013, Daniel Winz, Ervin Mazlagic

%This program is free software; you can redistribute it and/or
%modify it under the terms of the GNU General Public License
%as published by the Free Software Foundation; either version 2
%of the License, or (at your option) any later version.

%This program is distributed in the hope that it will be useful,
%but WITHOUT ANY WARRANTY; without even the implied warranty of
%MERCHANTABILITY or FITNESS FOR A PARTICULAR PURPOSE.  See the
%GNU General Public License for more details.
%----------------------------------------

\chapter{Rotation}
\section{Zentripetalkraft}
\[ F_z = \frac{m \cdot {v_b}^2}{r} = m \cdot \omega^2 \cdot r \]

\section{Rotationsbewegung}

\subsection{Winkel $\theta$}
\[ \theta =  \]

\subsection{Winkelgeschwindigkeit $\omega$}
\[ \omega = \frac{v_{tan}}{r} \]

\subsection{Winkelbeschleunigung $\alpha$}
\[ \alpha =  \]

\subsection{Drehmoment $M$}
\[ M = I \cdot \alpha = \vec{r} \times \vec{F} = r \cdot F \cdot \sin(\phi) 
= r \cdot F_{tan} \]

\subsection{Trägheitsmoment $I$}
Das Trägheitsmoment ist von der Form des Körpers abhängig. 
\[ I = \sum (m_i \cdot r^2) = \int(r^2)dm\]
Für einige einfache Körper existieren Formeln für die Berechnung des 
Trägheitsmoments. 
\begin{table}[h!]
\begin{tabular}{p{0.5\textwidth}l}
\rowcolor{lgray}
Körper 
& Trägheitsmoment \\

\rowcolor{white}
Stab mit Drehachse in der Mitte, senkrecht zur Symmetrieachse 
& $\dfrac{1}{12} \cdot m \cdot L^2$ \\

\rowcolor{lgray}
Stab mit Drehachse am Ende, senkrecht zur Symmetrieachse 
& $\dfrac{1}{3} \cdot m \cdot L^2$ \\

\rowcolor{white}
Platte mit Drehachse durch Zentrum, senkrecht zur Oberfläche
& $\dfrac{1}{12} \cdot m \cdot (a^2 + b^2)$ \\

\rowcolor{lgray}
Platte mit Drehachse entlang der Kante b
& $\dfrac{1}{3} \cdot m \cdot a^2$ \\

\rowcolor{white}
Dickwandiger Zylinder
& $\dfrac{1}{2} \cdot m \cdot ({r_1}^2 + {r_2}^2)$ \\

\rowcolor{lgray}
Vollzylinder
& $\dfrac{1}{2} \cdot m \cdot r^2$ \\

\rowcolor{white}
Dünnwandiger Zylinder
& $\dfrac{1}{} \cdot m \cdot r^2$ \\

\rowcolor{lgray}
Vollkugel
& $\dfrac{2}{5} \cdot m \cdot r^2$ \\

\rowcolor{white}
Hohlkugel
& $\dfrac{2}{3} \cdot m \cdot r^2$ \\
\end{tabular}
\end{table}

\subsection{Drehimpuls $L$}
\[ L = I \cdot \omega \]

\subsection{Rotationsenergie}
\[ E_{rot} = \frac{1}{2} \cdot I \cdot \omega^2 \]

\subsection{Leistung}
\[ P_{rot} = M \cdot \omega \]
