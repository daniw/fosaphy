% coding:utf-8

%----------------------------------------
%FOSAPHY, a LaTeX-Code for a summary of basic physics
%Copyright (C) 2013, Daniel Winz, Ervin Mazlagic

%This program is free software; you can redistribute it and/or
%modify it under the terms of the GNU General Public License
%as published by the Free Software Foundation; either version 2
%of the License, or (at your option) any later version.

%This program is distributed in the hope that it will be useful,
%but WITHOUT ANY WARRANTY; without even the implied warranty of
%MERCHANTABILITY or FITNESS FOR A PARTICULAR PURPOSE.  See the
%GNU General Public License for more details.
%----------------------------------------

\chapter{SI Einheiten}

Das Internationale Einheitensystem, SI genannt 
(franz. \textit{Système international d'unités}),
ist ein kohärentes dezimales und metrisches Einheitssystem für 
physikalische Grössen. Dieses wurde seit Ende des 18. Jahrhunderts
entwickelt und ist mittlerweile von jedem Staat offiziell eingeführt mit 
Ausnahme der USA, Myanmar und Liberia.

\newpage
\section{Grundeinheiten}
\begin{footnotesize}
\begin{tabular}{llll}
  \rowcolor{white} \textbf{Basisgrösse} & \textbf{Symbol} 
                    & \textbf{Einheit} & \textbf{Zeichen}\\
  \rowcolor{lgray} Länge       & $l$   & Meter     & $m$\\
  \rowcolor{white} Zeit        & $t$   & Sekunde   & $s$\\
  \rowcolor{lgray} Masse       & $m$   & Kilogramm & $kg$\\
  \rowcolor{white} Temperatur  & $T$   & Kelvin    & $K$\\
  \rowcolor{lgray} Stromstärke & $I$   & Ampere    & $A$\\
  \rowcolor{white} Stoffmenge  & $n$   & Mol       & $mol$\\
  \rowcolor{lgray} Lichtstärke & $I_v$ & Candela   & $cd$\\
\end{tabular}
\end{footnotesize}

\section{abgeleitete Einheiten}
\begin{footnotesize}
\begin{longtable}{p{0.35\columnwidth}lll}
  \rowcolor{white}  \textbf{Grösse}
                    & \textbf{Einheit}
                    & \textbf{Zeichen}
                    & \textbf{SI-Basiseinheiten} \\
  \rowcolor{lgray}  ebener Winkel
                    & Radiant
                    & $rad$
                    & $1$ \\
  \rowcolor{white}  räumlicher Winkel
                    & Steradiant
                    & $sr$
                    & $1$ \\
  \rowcolor{lgray}  Frequenz
                    & Hertz
                    & $Hz$
                    & $s^{-1}$ \\
  \rowcolor{white}  Kraft
                    & Newton
                    & $N$
                    & $m~kg~s^{-2}$ \\
  \rowcolor{lgray}  Druck, Spannung
                    & Pascal
                    & $Pa$
                    & $\frac{N}{m^2} = m^{-1}~kg~s^{-2}$ \\
  \rowcolor{white}  Energie, Arbeit, Wärmemenge
                    & Joule
                    & $J$
                    & $N~m = m^2~kg~s^{-2}$ \\
  \rowcolor{lgray}  Leistung, Energiestrom
                    & Watt
                    & $W$
                    & $\frac{J}{s} = m^2~kg~s^{-3}$ \\
  \rowcolor{white}  Elektrische Ladung, Elektrizitätsmenge
                    & Coulomb
                    & $C$
                    & $s~A$ \\
  \rowcolor{lgray}  elektrische Spannung, elektromotorische Kraft
                    & Volt
                    & $V$
                    & $\frac{W}{A} = m^2~kg~s^{-3}~A^{-1}$ \\
  \rowcolor{white}  elektrische Kapazität
                    & Farad
                    & $F$
                    & $\frac{C}{V} = m^{-2}~kg^{-1}~s^4~A^2$ \\
  \rowcolor{lgray}  elektrischer Widerstand
                    & Ohm
                    & $\Omega$
                    & $\frac{V}{A} = m^2~kg~s^{-3}~A^{-2}$ \\
  \rowcolor{white}  elektrischer Leitwert
                    & Siemens
                    & $S$
                    & $\frac{A}{V} = m^{-2}~kg{-1}~s^3~A^2$ \\
  \rowcolor{lgray}  magnetischer Fluss
                    & Weber
                    & $Wb$
                    & $V~s = m^2~kg~s^{-2}~A^{-1}$ \\
  \rowcolor{white}  magnetische Flussdichte
                    & Tesla
                    & $T$
                    & $\frac{Wb}{m^2} = kg~s^{-2}~A^{-1}$ \\
  \rowcolor{lgray}  Induktivität
                    & Henry
                    & $H$
                    & $\frac{Wb}{A} = m^2~kg~s^{-2}~A^{-2}$ \\
  \rowcolor{white}  Celsius-Temperatur
                    & Grad Celsius
                    & $^{\circ} C$
                    & $K$ \\
  \rowcolor{lgray}  Lichtstrom
                    & Lumen
                    & $lm$
                    & $cd~sr = cd$ \\
  \rowcolor{white}  Beleuchtungsstärke
                    & Lux
                    & $lx$
                    & $\frac{lm}{m^2} = m^{-2}~cd$ \\
  \rowcolor{lgray}  Aktivität eines Radionuklids
                    & Becquerel
                    & $Bq$
                    & $s^{-1}$ \\
  \rowcolor{white}  Energiedosis, spezifische übertragene Energie, Kerma
                    & Gray
                    & $Gy$
                    & $\frac{J}{kg} = m^2~s^{-2}$ \\
  \rowcolor{lgray}  Äquivalentdosis, Umgebungsäquivalentdosis, 
                    Richtungsäquivalentdosis, Richtungsäquivalentdosis
                    & Sievert
                    & $Sv$
                    & $\frac{J}{kg} = m^2~s^{-2}$ \\
  \rowcolor{white}  Katalytische Aktivität
                    & Katal
                    & $kat$
                    & $s^{-1}~mol$ \\
\end{longtable}
\end{footnotesize}

\section{nicht-SI Einheiten}
\begin{footnotesize}
\begin{longtable}{p{0.25\columnwidth}lll}
  \rowcolor{white}  \textbf{Grösse}
                    & \textbf{Einheit}
                    & \textbf{Zeichen}
                    & \textbf{SI} \\
  \rowcolor{lgray}  Leistung
                    & Pferdestärken
                    & $PS$
                    & $1PS = 745.7W$ \\
  \rowcolor{white}  Länge
                    & mil
                    & $mil$
                    & $1mil = 25.4\mu m$ \\
  \rowcolor{lgray}  Länge
                    & Zoll / Inch
                    & $In$ / $''$
                    & $1'' = 0.0254m$ \\
  \rowcolor{white}  Länge
                    & Fuss
                    & $'$
                    & $1' = 0.3048m$ \\
  \rowcolor{lgray}  Länge
                    & Meile
                    & $$
                    & $1\text{Meile} = 1609.344m$ \\
  \rowcolor{white}  Länge
                    & Seemeile
                    & $$
                    & $1\text{Seemeile} = 1852m$ \\
  \rowcolor{lgray}  Energie
                    & Kalorie
                    & $kal$
                    & $1kal = 4.1868J$ \\
  \rowcolor{white}  Geschwindigkeit
                    & Kilometer pro Stunde
                    & $km/h$
                    & $3.6km/h = 1m/s$ \\
  \rowcolor{lgray}  Temperatur
                    & Celsius
                    & $^\circ C$
                    & $\vartheta_C = \vartheta_K + 273.15$ \\
%   \rowcolor{white}  Temperatur
%                     & Farenheit
%                     & $^\circ F$
%                     & $\vartheta_F = \frac{9}{5} \cdot \vartheta_K + 255.372$ \\
\end{longtable}
\end{footnotesize}
