% coding:utf-8

%----------------------------------------
%FOSAPHY, a LaTeX-Code for a summary of basic physics
%Copyright (C) 2013, Daniel Winz, Ervin Mazlagic

%This program is free software; you can redistribute it and/or
%modify it under the terms of the GNU General Public License
%as published by the Free Software Foundation; either version 2
%of the License, or (at your option) any later version.

%This program is distributed in the hope that it will be useful,
%but WITHOUT ANY WARRANTY; without even the implied warranty of
%MERCHANTABILITY or FITNESS FOR A PARTICULAR PURPOSE.  See the
%GNU General Public License for more details.
%----------------------------------------

\chapter{Schwingung}
\section{Einfache harmonische Schwingung}
Rücktreibende Kraft: 
\[ \boxed{F = - k \cdot x} \]
\begin{tabular}{ll}
$F:$ & Rücktreibende Kraft \\
$k:$ & Federkonstante \\
$x:$ & Auslenkung
\end{tabular}
\[ \boxed{F = m \cdot a(t) = m \cdot \frac{d^2}{d t^2}x(t) = m \cdot \ddot{x}} \]
\[ \boxed{m \cdot \ddot{x} + k \cdot x = 0} \]
\[ \boxed{x(t) = A \cdot \cos(\omega \cdot t + \phi)} \]
\[ \boxed{\omega = 2 \cdot \pi \cdot f = \sqrt{\frac{k}{m}}} \]
\[ \boxed{T = 2 \cdot \pi = \sqrt{\frac{m}{k}}} \]
\[ \boxed{\dot{x} = - A \cdot \omega \cdot \sin(\omega \cdot t + \phi)} \]
\[ \boxed{\ddot{x} = - A \cdot \omega^2 \cdot \cos(\omega \cdot t + \phi)} \]
\[ \boxed{m \cdot (-A \cdot \omega^2 \cdot \cos(\omega \cdot t + \phi)) 
+ k \cdot A \cdot \cos(\omega \cdot t + \phi) \stackrel{!}{=} 0} \]
\[ \boxed{A \cdot \cos(\omega \cdot t + \phi) \cdot (-m \cdot  \omega^2 + k) 
\stackrel{!}{=} 0} \]
\[ \boxed{-m \cdot  \omega^2 + k \stackrel{!}{=} 0} \]
\[ \boxed{\omega^2 = \frac{k}{m}} \]
\[ \boxed{\omega = \sqrt{\frac{k}{m}}} \]

\subsection{Auslenkung und Geschwindigkeit als Anfangsbedingungen}
\[ \boxed{x_0 = x(0) = A \cdot \cos(\phi)} \]
\[ \boxed{v_0 = \dot{x}(0) = - \omega \cdot A \cdot \sin(\phi)} \]
\[ \boxed{\phi = \tan^{-1}\left(-\frac{v_0}{\omega \cdot x_0}\right)} \]
\textbf{Achtung!} Lösung kann im falschen Quadranten liegen. 
Evtl. mit $pi$ addieren. 
\[ \boxed{A = \sqrt{{x_0}^2 + \left(\frac{v_0}{\omega}\right)^2}} \]

\subsection{Geschwindigkeit und Beschleunigung aus Amplitude und Position}
\[ \boxed{v(t) = \sqrt{\frac{k}{m}} \cdot \sqrt{A^2 - x^2(t)}} \]
\[ \boxed{a(t) = - \frac{k}{m} \cdot x(t)} \]