% coding:utf-8

%----------------------------------------
%FOSAPHY, a LaTeX-Code for a summary of basic physics
%Copyright (C) 2013, Daniel Winz, Ervin Mazlagic

%This program is free software; you can redistribute it and/or
%modify it under the terms of the GNU General Public License
%as published by the Free Software Foundation; either version 2
%of the License, or (at your option) any later version.

%This program is distributed in the hope that it will be useful,
%but WITHOUT ANY WARRANTY; without even the implied warranty of
%MERCHANTABILITY or FITNESS FOR A PARTICULAR PURPOSE.  See the
%GNU General Public License for more details.
%----------------------------------------

\chapter{Schwingung}
\section{Einfache harmonische Schwingung}
Rücktreibende Kraft: 
\[ \boxed{F = - k \cdot x} \]
\begin{tabular}{ll}
$F:$ & Rücktreibende Kraft \\
$k:$ & Federkonstante \\
$x:$ & Auslenkung
\end{tabular}
\[ \boxed{F = m \cdot a(t) = m \cdot \frac{d^2}{d t^2}x(t) = m \cdot \ddot{x}} \]
\[ \boxed{m \cdot \ddot{x} + k \cdot x = 0} \]
\[ \boxed{x(t) = A \cdot \cos(\omega \cdot t + \phi)} \]
\[ \boxed{\omega = 2 \cdot \pi \cdot f = \sqrt{\frac{k}{m}}} \]
\[ \boxed{T = 2 \cdot \pi = \sqrt{\frac{m}{k}}} \]
\[ \boxed{\dot{x} = - A \cdot \omega \cdot \sin(\omega \cdot t + \phi)} \]
\[ \boxed{\ddot{x} = - A \cdot \omega^2 \cdot \cos(\omega \cdot t + \phi)} \]
\[ \boxed{m \cdot (-A \cdot \omega^2 \cdot \cos(\omega \cdot t + \phi)) 
+ k \cdot A \cdot \cos(\omega \cdot t + \phi) \stackrel{!}{=} 0} \]
\[ \boxed{A \cdot \cos(\omega \cdot t + \phi) \cdot (-m \cdot  \omega^2 + k) 
\stackrel{!}{=} 0} \]
\[ \boxed{-m \cdot  \omega^2 + k \stackrel{!}{=} 0} \]
\[ \boxed{\omega^2 = \frac{k}{m}} \]
\[ \boxed{\omega = \sqrt{\frac{k}{m}}} \]

\subsection{Auslenkung und Geschwindigkeit als Anfangsbedingungen}
\[ \boxed{x_0 = x(0) = A \cdot \cos(\phi)} \]
\[ \boxed{v_0 = \dot{x}(0) = - \omega \cdot A \cdot \sin(\phi)} \]
\[ \boxed{\phi = \tan^{-1}\left(-\frac{v_0}{\omega \cdot x_0}\right)} \]
\textbf{Achtung!} Lösung kann im falschen Quadranten liegen. 
Evtl. mit $pi$ addieren. 
\[ \boxed{A = \sqrt{{x_0}^2 + \left(\frac{v_0}{\omega}\right)^2}} \]

\subsection{Geschwindigkeit und Beschleunigung aus Amplitude und Position}
\[ \boxed{v(t) = \sqrt{\frac{k}{m}} \cdot \sqrt{A^2 - x^2(t)}} \]
\[ \boxed{a(t) = - \frac{k}{m} \cdot x(t)} \]

\subsection{Maximale Geschwindigkeit und Beschleunigung}
\[ \boxed{v_{max} = \omega \cdot A} \]
\[ \boxed{v_{min} = -\omega \cdot A} \]
\[ \boxed{a_{max} = -\omega^2 \cdot A} \]

\subsection{Vertikales Federpendel}
\[ \boxed{\omega = \sqrt{\frac{k}{t}}} \]
\[ \boxed{x(t) = - \frac{m \cdot g}{k} + A \cdot \cos(\omega \cdot t + \phi)} \]

\subsection{Energie}
\[ \boxed{E_{pot} = \frac{1}{2} \cdot k \cdot x(t)^2} \]
\[ \boxed{E_{kin} = \frac{1}{2} \cdot m \cdot v(t)^2} \]

\section{Rotationsschwingung}
\[ \boxed{\omega = \sqrt{\frac{\kappa}{I}}} \]
\[ \boxed{\Theta(t) = \Theta_0 \cdot \cos(\omega \cdot t + \phi)} \]

\subsection{Stab am Ende aufgehängt}
\[ \boxed{I = \frac{1}{12} \cdot m \cdot L^2} \]
\[ \boxed{M = -F \cdot r \cdot \sin(\angle) = -k \cdot \Delta x \cdot \frac{L}{2}} \]
\[ \boxed{M = -k \left(\frac{L}{2}\right)^2 \cdot \Theta} \]
Für kleine Winkel: 
\[ \boxed{\kappa = k \cdot \left(\frac{L}{2}\right)} \]
\[ \boxed{I = \frac{1}{12} \cdot m \cdot L^2} \]
\[ \boxed{\omega = \sqrt{\frac{\kappa}{I}} 
= \sqrt{\frac{k \cdot \left(\frac{L}{2}\right)^2}{\frac{1}{12}\cdot m \cdot L^2}} 
= \sqrt{\frac{3 \cdot \kappa}{m}}} \]

\section{Physikalisches Pendel}
\[ \boxed{M = -d \cdot m \cdot g \cdot \sin(\Theta) 
\cong - \underbrace{d \cdot m \cdot g}_{\kappa} \cdot \Theta} \]
\[ \boxed{\omega = \sqrt{\frac{\kappa}{I}} 
= \sqrt{\frac{m \cdot g \cdot d}{I_z}}} \]
speziell: Fadenpendel
\[ \boxed{I = m \cdot L^2} \]
\[ \boxed{\kappa = m \cdot g \cdot L} \]
\[ \boxed{2 \cdot \pi \cdot f = \sqrt{\frac{m \cdot g \cdot L}{m \cdot L^2}}} \]
\[ \boxed{f = \frac{1}{2 \cdot \pi} \cdot \sqrt{\frac{g}{L}}} \]
Fadenkraft: 
\[ \boxed{F_s = m \cdot g \cdot \cos(\theta) + m \cdot \frac{v^2}{L}} \]

\section{Pendelnde Flüssigkeitssäule}
\[ \boxed{omega = \sqrt(\frac{2 \cdot g}{L})} \]

\section{Gedämpfte Schwingung}
Fall 1: $\beta > \omega$ "'Kriechfall"'
\[ \boxed{x(t) \sim e^{(-\beta \pm \delta)t}} \]
Fall 3: $\omega > \beta$ "'Gedämpfte Schwingung"'
\[ \boxed{x(t) = A \cdot e^{-\beta t} \cdot \cos(\omega_d t)} \]
\[ \boxed{\omega_d = \sqrt{\omega^2 - \beta^2}} \]
Fall 2: $\omega = \beta$ "'kritische Dämpfung"'
\[ \boxed{\beta = \frac{b}{2 m} \quad \Leftrightarrow \quad 
b = b_{krit} = \sqrt{4 \cdot k \cdot m}} \]
Zerfallszeit: 
\[ \boxed{\tau = \frac{1}{\beta}} \]
\[ \boxed{A(t) = A \cdot e^{-\beta t} = A \cdot e^{-\frac{t}{\tau}}} \]
Bei der Zeit $t = \tau$
\[ \boxed{A(\tau) = \frac{A}{e} = A \cdot e^{-1} \approx 0.37 A} \]
Abklingkonstante: 
\[ \boxed{\beta 
= \frac{1}{t_2 - t_1} \cdot \ln\left(\frac{x(t_1)}{x(t_2)}\right)} \]
Schwingungsenergie: 
\[ \boxed{E(t) = \frac{m}{2} \cdot \dot{x}^2(t) + \frac{k}{2} \cdot x^2(t)} \]
\[ \boxed{\frac{d E(t)}{dt} 
= \frac{m}{2} \cdot 2 \cdot \dot{x}^2(t) \cdot \ddot{x}^2(t) 
+ \frac{k}{2} \cdot 2 \cdot x^2(t) \cdot \dot{x}^2(t) 
= \underbrace{(m \ddot{x}(t) + k x(t))}_{-b \dot{x}(t)} \cdot \dot{x}(t)} \]
\[ \boxed{\frac{d E(t)}{dt} = -b \dot{x}^2 - b v^2(t) \leq 0} \]
\[ \boxed{A(t) = A \cdot e^{-\frac{t}{\tau}}} \]
\[ \boxed{E(t) = E_0 \cdot e^{-\frac{t}{\frac{\tau}{2}}}} \]
Güte: 
\[ \boxed{Q = \frac{\pi}{\beta \cdot T_d} = \pi \cdot \frac{\tau}{T_d} 
= \frac{\omega_d \cdot \tau}{2}} \]

\section{Erzwungene Schwingung}
\[ \boxed{x(t) = A(\Omega) \cdot \cos(\Omega t - \varphi(\Omega))} \]
\[ \boxed{A(\Omega) = \frac{F_0}{m \cdot 
\sqrt{(\omega^2 - \Omega^2)^2 + (2 \cdot \Omega \cdot \beta)^2}} 
\approx \frac{H}{\sqrt{\left(1 - \left(\frac{\Omega}{\omega}\right)^2\right)^2 
+ \left(\frac{\Omega}{Q \cdot \omega}\right)^2}}} \]  
\[ \boxed{\varphi(\Omega) = \arctan\left(\frac{2 \cdot \Omega \cdot \beta}
{{\omega_0}^2 - \Omega^2}\right) 
= \arctan{\left(\frac{\frac{b}{\sqrt{k \cdot m}} 
\left(\frac{\Omega}{\omega}\right)}
{1 - \left(\frac{\Omega}{\omega}\right)^2}\right)}} \]
\[ \boxed{H = \frac{F}{k}} \]
\[ \boxed{\omega = \sqrt{\frac{k}{m}}} \]
\[ \boxed{\beta = \frac{b}{2 m} = \frac{1}{\tau}} \]
\[ \boxed{Q = \frac{\omega_d}{2\beta} = \omega_d \frac{m}{b} 
= \pi \frac{\tau}{T}} \]
schwache Dämpfung: 
\[ \boxed{b << b_{krit} = \sqrt{4 k \cdot m} \quad Q >> 1} \]

\subsection{Resonanzfrequenz}
\[ \boxed{\frac{d A(\Omega)}{d\Omega} = 0} \]
\[ \boxed{\Omega_R = \sqrt{\omega^2 - 2\beta^2} \neq \omega_d 
= \sqrt{\omega^2 - \beta^2} \neq \omega} \]
\[ \boxed{A(\Omega_R) = \frac{\omega^2 H}{2 \beta \sqrt{\omega^2 - \beta^2}} 
= \frac{k H}{m \sqrt{(\omega^2 - {\Omega_R}^2)^2 + (2\beta\Omega_R)^2}} 
\approx \frac{Q H}{\sqrt{1 - \frac{1}{2 Q^2}}} \approx Q \cdot H} \]
\[ \boxed{\varphi_R = \varphi(\Omega_R) 
= \arctan\left(\frac{2 \beta \omega_R}{\omega^2 - {\Omega_R}^2}\right) 
\approx \arctan\left(\sqrt{4 Q^2 - 2}\right) \approx \frac{\pi}{2}} \]

% ohne Dämpfung ($b = 0$): 
% \[ \boxed{A(\Omega) = \frac{F_0}{m \cdot \sqrt{(\omega^2 - \Omega^2)^2}} 
% = \frac{\omega^2 \cdot H}{|\omega^2 - \Omega^2|}} \]  
% \[ \boxed{\varphi(\Omega) = \arctan\left(\frac{b \cdot \Omega}
% {m \cdot (\omega^2 - \Omega^2)}\right)} \]
