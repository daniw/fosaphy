% coding:utf-8

%----------------------------------------
%FOSAPHY, a LaTeX-Code for a summary of basic physics
%Copyright (C) 2013, Daniel Winz, Ervin Mazlagic

%This program is free software; you can redistribute it and/or
%modify it under the terms of the GNU General Public License
%as published by the Free Software Foundation; either version 2
%of the License, or (at your option) any later version.

%This program is distributed in the hope that it will be useful,
%but WITHOUT ANY WARRANTY; without even the implied warranty of
%MERCHANTABILITY or FITNESS FOR A PARTICULAR PURPOSE.  See the
%GNU General Public License for more details.
%----------------------------------------

\chapter{Wellen}
\section{Allgemein}
\[ \boxed{f(x,t) = f^*(x \pm v \cdot t} \]

\section{Seilwelle}
\begin{tabular}{@{}ll}
$FS$:   & Seilkraft in $kg \frac{m}{s^2}$ \\
$\mu$:  & Längendichte in $\frac{kg}{m}$ \\
$v$:    & Geschwindigkeit
\end{tabular}
Winkel klein: $\cos\phi \approx 1 \quad \sin\phi \approx \phi \approx \tan\phi$
\[ \boxed{-Fs \sin \phi_L + F_s \sin \phi_R 
= F_s (\frac{\partial}{\partial x} f(x,t) 
- \frac{\partial}{\partial x} f(x-\Delta x,t) )} \]
Wellengleichung: 
\[ \boxed{\frac{F_s}{\mu} \frac{\partial^2}{\partial x^2} f(x,t) 
= \frac{\partial^2}{\partial t^2} f(x,t)} \]
Gag: Jede Funktionen der Form $f(x,t) = f^*(x \pm v t)$ ist eine Lösung
\[ \boxed{v = \sqrt{\frac{F_s}{\mu}}} \]
- nach rechts, + nach links
\[ \boxed{t = \frac{s}{v}} \]
Anfangsbedingung (rechtslaufende Welle): 
\[ \boxed{f(x,0) = f^*(x) = A \cos(\frac{2 \pi}{\lambda} x + \phi)} \]
Allgemeine Lösung (rechtslaufende Welle): 
\[ \boxed{f(x,t) = f^*(x-vt) = A \cos (\frac{2 \pi}{\lambda}(x-vt)+\phi)} \]
\[ \boxed{f(x,t) = A \cos(kx-\omega t + \phi)} \]
\[ \boxed{\omega = \frac{2 \pi}{T}} \]
\[ \boxed{k = \frac{2 \pi}{\lambda} \quad \text{Wellenzahl}} \]
\[ \boxed{v = \frac{\lambda}{T} = \frac{\omega}{k}} \]
%\[ \boxed{} \]

\section{Druckwellen, Schallwellen}
Fluid: 
\[ \boxed{v = \sqrt{\frac{1}{\kappa \cdot \rho}}} \]
Kompressibilität: 
\[ \boxed{\kappa = -\frac{1}{V} \frac{d V}{d \rho}} \]
Festkörper: 
\[ \boxed{\sqrt{\frac{E}{\rho}}} \]
\begin{tabular}{ll}
Luft (20C) & 343 m/s \\
Helium (20C) & 999 m/s \\
Wasserstoff (20C) & 1330 m/s \\
flüssiges Helium (4k) & 211 m/s \\
Wasser (0C) & 1402 m/s \\
Wasser (20C) & 1482 m/s \\
Quecksilber (20C) & 1451 m/s \\
Aluminium & 6420 m/s \\
Blei & 1960 m/s \\
Stahl & 5941 m/s \\
\end{tabular}

\section{Energiefluss}
\[ \boxed{P \propto A^2} \]
\subsection{Energiefluss der harmonischen Seilwelle}
\[ \boxed{P_{Seil} = \frac{1}{2} \mu v \cdot \omega^2 A^2 
= \frac{1}{2} \rho S v \omega^2 A^2} \]
\subsection{Energiefluss einer Schallwelle}
\[ \boxed{\hat{p} = \rho \omega v A} \qquad \text{Schalldruck} \]
\[ \boxed{P_{Schall} = \frac{1}{2} \rho S v \omega^2 A^2 
= \frac{1}{2} \frac{S}{\rho v} \hat{p}^2} \]

\section{Intensität einer Welle}
\subsection{Ebene Welle oder gerichteter Strahl}
\[ \boxed{I = \frac{P_{av}}{S}} \qquad S: \text{Querschnittsfläche}\]
\[ \boxed{I_{Seil} = \frac{1}{2} \rho v \omega^2 A^2} \]
\[ \boxed{I_{Schall} = \frac{1}{2} \frac{\hat{p}^2}{\rho v}} \]
\subsection{Zylinder Welle oder Kreisfläche}
\[ \boxed{I(r) = \frac{P_{av}}{S_{Zylinder}} = \frac{P_{av}}{L \cdot 2 \pi r} 
\propto \frac{1}{r}} \]
\subsection{Kugelwelle}
\[ \boxed{I(r) = \frac{P_{av}}{S_{Kugelschale}} = \frac{P_{av}}{4 \pi r^2} 
\propto \frac{1}{r^2}} \]

\section{Schallpegel}
\[ \boxed{L = 10 \cdot \log\left(\frac{I}{I_0}\right)} 
\qquad I_0 = 10^{-12} \frac{W}{m^2} \text{(Hörschwelle)} \]

\section{Dopplereffekt}
Bewegter Sender: 
\[ \boxed{f' = \frac{v}{\lambda \pm v_S T} = \frac{v}{v \pm v_S}f} \]
Bewegter Empfänger: 
\[ \boxed{f' = \frac{v \pm v_E}{\lambda} = \frac{v \pm v_E}{v}f} \]
Kombiniert: 
\[ \boxed{f' = \frac{v \pm v_E}{v \pm v_S}f} \qquad \substack{\text{
$+$, wenn e sich auf Quelle zu bewegt. }\\\\\text{
$+$, wenn s sich vom Empfänger weg bewegt. }} \]

\subsection{Machscher Kegel}
\[ \boxed{\sin(\theta) = \frac{v t}{v_S t} = \frac{v}{V_s} = \frac{1}{Ma}} \]

\section{Schwebung}
\[ \boxed{2 \pi \cdot f_B = |\omega_1 - \omega_2|} \]
\[ \boxed{2 \pi \cdot f_{av} = \frac{\omega_1 + \omega_2}{2}} \]

\section{Reflektion}
\{ \boxed{P_{in} = R_r + P_t} \}
\{ \boxed{A_{in} + A_r = A_t} \}
Achtung Vorzeichen!

