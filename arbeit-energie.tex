\chapter{Arbeit und Energie}

\section{Arbeit}
Arbeit ist ein Ausdruck der verwendet wird um eine mechanische 
Energieübertrgung zu beschreiben. Diese beschreibt das Produkt aus
Kraft und Weg, wobei diese vektoriell sind.
\[ \boxed{ W = \vec{F} \cdot \vec{s} 
	= ||\vec{F}|| \cdot ||\vec{s}|| \cdot 
	\sphericalangle \left( \vec{F}, \vec{s} \right) } 
	\qquad \left[ J = N \cdot m = \frac{kg \cdot m^2}{s^2} \right] \]

\subsection{Goldene Regel der Mechanik}
Die goldene Regel der Mechanik besagt, dass man Arbeit als konstantes 
Produkt aus Kraft und Weg betrachten kann. Dies ist die Definition des
Energieerhaltungssatzes für den Bereich der Mechanik (z.B. einfache
Machienen wie Flaschenzüge). 
\[ \boxed{W = \vec{F} \cdot \vec{s}} \] 
Dies bedeutet, man kann für die gleiche Arbeit jeweils einen verschieden
langen Weg nehmen. So kann die hierzu nötige Kraft variiert werden. 
Wichtig bei der Anwendung dieser goldenen Regel ist, dass der Weg als
auch die Kraft als vekorielle Grössen verstanden werden. 

\section{Energie}
Die Energie ist eine fundamentale Grösse aller Teilbereiche der Physik.
Diese kann in verschiedenen Formen auftreten. In der Mechanik werden vor
allem die potentielle und kinetische Energie behandelt.

\subsection{Potentielle Energie}
Die potentielle Energie ist definiert als jene Energie, welche sich aus
einem Höhenunterschied (Potentialdifferenz) einer Masse zu einer 
Referenzhöhe ergibt. 
\[ \boxed{ E_{Pot} = m \cdot g \cdot (h - h_{Ref}) } \]

\subsection{Kinetische Energie}
Die kinetische Energie ist proportional zur Masse des sich bewegenden Körpers
und quadratisch zu deren Geschwindigkeit. Wichtig ist hierbei auch die 
Betrachtung der Bewegung des Körpers, vor allem zu welcher Referez die 
Geschwindigkeit gilt (Relativgeschwindugkeit).
\[ \boxed{ E_{Kin_{Translation}} = \frac{m \cdot \vec{v}^2}{2} } \]
Die kinetische Energie kann aber auch eine Rotation beschreiben.
\[ \boxed{ E_{Kin_{Rotation}} = \frac{\vec{J} \cdot \omega^2}{2} } \]

\subsection{Federenergie}
Die Federengergie ist definiert als das Produkt aus Federkonstante $k$ und 
dem Integral der Dehnung bzw. des Weges $s$ 
(siehe Kapitel \ref{sec:feder-energie}).
\[ \boxed{E_{Feder} = k \cdot \int_{s_a}^{s_b} s \cdot ds = \frac{k \cdot s^2}{2}} \]
